\documentclass[12]{amsart}
\usepackage{latexsym,amssymb}
\usepackage{graphics}
\usepackage{harvard}
\usepackage{epsfig,subfigure}
\renewcommand{\theequation}{\thesection.\arabic{equation}}

\begin{document}
\bibliographystyle{econometrica}

\title{Fragments on GLMX}

\thanks{Version: \today.}  

\author{Roger Koenker}
\author{Achim Zeileis}
%\address{University of Illinois at Urbana-Champaign}

\maketitle

\section{Some Simulation Results}
To explore performance of the estimators described above, we have undertaken a small
simulation experiment.  Data is generated from the simple latent variable model,
\[
Y_i = \beta_0 + x_i \beta_1 + (1 + x_i \gamma ) u_i
\]
with design points, $x_i$ iid $U[1,10]$ and $u_i$ iid standard logistic.  Given
$n$ observations from this model, we compute the four quintiles of the unconditional
response, $y \in \{ Y_{(in/5)}, \; i = 1, \cdots 4 \}$, and fit the model:
\begin{verbatim}
    fit <- hetglm(  ( Y > y )  ~ x, 
        family = binomial(link= ``logit''), link.scale = ``identity'')
\end{verbatim}
Letting $F$ denote the standard logistic distribution function, and using
its symmetry, we have,
\begin{eqnarray}
	P(Y_i > y) &=& 1 - F( ( y - \beta_0 - x_i \beta_1 )/(1 + x_i \gamma))\\
	&=& F( ( \beta_0 + x_i \beta_1 - y)/(1 + x_i \gamma)),
\end{eqnarray}
so our maximum likelihood estimator is intended to estimate the vector
$(\beta_0 , \beta_1 , \gamma)$.  Since $y$ varies from sample to sample
and is obviously a quantity known to the investigator, we recenter the
estimated intercept by adding back $y$, giving an estimator
$(\hat \beta_0 , \hat \beta_1 , \hat \gamma)$.  We take
$(\beta_0 , \beta_1 , \gamma) = (0, 3, 0.25)$, and consider sample sizes
$n \in \{ 500, 1000, 5000 \}$.

In Table \ref{Tab1} we report median bias and  median absolute errors for
the simulation based on 1000 replications.  It is evident that the performance
at $n = 500$ is rather poor, but at $n= 5000$ there is considerable improvement.
Corresponding results for mean bias and root mean squared error appear in
Table \ref{Tab2}, but should not be viewed by the weak at heart.

\input tab1
\input tab2

\end{document}
